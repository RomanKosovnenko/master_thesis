\RequirePackage[ngerman=ngerman-x-latest]{hyphsubst}
\documentclass[ngerman,twoside]{tudscrreprt}
%\usepackage{selinput}\SelectInputMappings{adieresis={ä},germandbls={ß}}
%\usepackage[T1]{fontenc}
\usepackage{babel}
\usepackage[utf8]{luainputenc}
\usepackage{isodate}
\usepackage{tudscrsupervisor}
\usepackage{enumitem}\setlist{noitemsep}

\begin{document}
\faculty{Fakultät Informatik}\department{Institut für Software- und Multimediatechnik}
%\institute{Institut für Kriminologie}
\chair{Lehrstuhl für Softwaretechnik}
\title{%
  Das ist der Titel
}
\thesis{bachelor}
%\graduation[M.Sc.]{Master of Science}
\author{%
  Firstname Lastname\matriculationnumber{123456}
  \dateofbirth{01.01.1996}\placeofbirth{Dresden}
  \discipline{Bachelor Informatik 2009}
}%\matriculationyear{1234}
\issuedate{01.08.2018}\duedate{01.11.2018}
\supervisor{%
  Dipl.-Inf. Carl Mai \and
  Dr.\ Thomas Kühn
}
\professor{Prof.\ Dr.\ rer.\ nat.\ habil. Uwe Aßmann}

\taskform{%
  In dieser Arbeit soll untersucht werden. blabla [1].

  \textbf{Fragestellungen:}
  \begin{itemize}
    \item blabla
    \item blabla
  \end{itemize}

  \textbf{Aus den Fragestellungen lassen sich folgende Aufgaben ableiten:}
  \begin{itemize}
    \item	blabla
  \end{itemize}

\textbf{Quellen:}
\begin{itemize}
  \item	[1] Adaptive Petri Nets - A Petri Net Extension for Reconfigurable Structures (Carl Mai, René Schöne, Johannes Mey, Thomas Kühn and Uwe Aßmann), In Adaptive 2018
\end{itemize}

}{%
}

\end{document}
