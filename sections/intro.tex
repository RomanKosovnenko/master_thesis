\chapter{Introduction}\label{intro}
Optimization problem occurs in all aspects of our life and have complex classification. There exist different methods to solve optimization problems. One of them is evolutionary computing~(EC) approach. The evolutionary algorithms~(EAs) are part of EC. EAs could solve many optimization problems. It is especially good when user does know additional information about the problem. All algorithms of EAs are based on variants operators and parameters. It means that parameter values could influence the solution of optimization problem. This thesis describes parameter tuning and parameter analysis of genetic optimization approach of solving a problem of software variant selection and hardware resource allocation.  

This chapter describes the motivation for parameter tuning and analysis of genetic optimization approach and marks out research questions to be answered in this thesis. It additionally outlines a solution and contains an overview of the thesis structure.

\section{Motivation}
Researchers and developers know that good parameter values could improve the results of the algorithm. But searching for optimized parameter values is another problem, that needs to be solved. This problem is one of the persisting challenges of the EC field~\cite{smit2010parameter}. The main problem is that specific EA contains unique design choices that presented in form of operators or its parameters. 

There exist many recommendations about what values are preferable~\cite{de2007parameter, sipper2018investigating}.  But in experimental researches~\cite{de2007parameter, shahookar1990genetic, gockel1997influencing}, these recommendations does not work. 

As a result, parameters need to be tuned. EAs have two approaches for parameter setting: parameter tuning and parameter control. Parameter control could change the parameter value during the EA work. Parameter tuning finds optimized values before EA starts to solve the optimization problem. In our case, this problem is a problem of software variant selection and hardware resource allocation. 

Due to fact, that genetic approach solves not all problems of mentioned problem type, like ILP~\cite{gotz18} approach does.
This thesis aims to overcome the gape between approaches and improve the genetic optimization approach.


\section{Objective}
The goal of this thesis is to improve a previously developed genetic optimization approach using parameter tuning.
The research objective is to identify and/or introduce parameters that can improve the genetic optimization approach in terms of its performance and scalability and the quality of the obtained solution. We need to answer the following research questions in order to reach the research objective: 
\begin{itemize}
	\item \textbf{RQ1}: Does the parameter tuning improve the results, and what effect does it give?
	\item \textbf{RQ2}: Were there any bad design choices in the genetic solver? Is there any way to improve it?
\end{itemize}

To answer these questions, we are using several methods that improve the genetic optimization approach. Moreover, we evaluate it on each stage. 

\section{Solution}

The described modifications of the genetic optimization approach use parameter tuning to obtain the optimized parameter values. Some of them introduce new parameters that influence the results in terms of scalability and quality or help to avoid trapping in a local optimum.

The evaluation shows that the presented solution gives a twofold increase in terms of the number of valid solutions.


\section{Overview}
This thesis is organized as follows: In Chapter~\ref{chapter:background}, we extend not advanced in the field of software selection and hardware resource allocation problem, evolutionary algorithms, and parameter tuning reader by background knowledge and defines the scope of the thesis. Chapter~\ref{chapter:Implementation} describes an iterative approach to improving the genetic approach. The evaluation results and analysis could be found in Chapter~\ref{chapter:evaluationAnalysis}. Finally, Chapter~\ref{chapter:conclusion} concludes the thesis and Chapter~\ref{chapter:futureWork} describe the future work.
