\chapter{Introduction}\label{intro}

Many optimization problems could be solved using evolutionary programming (e.g. Genetic Algorithm). 

\section{Motivation}
- 


\section{Objective}
The objective of this thesis is to improve a previously developed genetic optimization approach by carefully tuning its parameters. The research objective is to identify and/or introduce parameters that can significantly influence the quality of the obtained solution, improve performance and increase scalability of the genetic optimization approach. To reach this goal we need to answer two research questions:
\begin{itemize}
	\item \textbf{RQ1}: Does the parameter tuning improve the results and what effect does it give?
	\item \textbf{RQ2}: Were there any bad design choices in the genetic solver? Is there any way to improve it?
\end{itemize}

\section{Solution}


  	
\section{Overview}
This thesis is organized as follows: In Chapter 2, we extend not advanced in field of problem solving and optimizations reader by background knowledge and defines scope of thesis. Chapter 3 describes related work in defined scope. In Chapter 4 one will find the concept description of dynamic heuristics selection. Chapter 5 contains more detailed information about approach implementation and  embedding it to BRISE. The evaluation results and analysis could be found in Chapter 6. Finally, Chapter 7 concludes the thesis and Chapter 8 describe the future work.
