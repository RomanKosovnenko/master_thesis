\chapter{Future work}
\label{chapter:futureWork}

The research presented in this thesis shows that the Parameter Tuning could improve the Genetic Solver. However, it is not enough to solve big sized problems. As mentioned in Section~\ref{sec:GeneticAlgorithm:Selector}, there is a framework limitation of used selector algorithms. There is a possibility that another selection algorithm for EA could further improve results. Because during the tuning parameter, BRISE gives optimized parameters that differed only in the selector.

Genetic solver without duplicates~(WD) is a possible starting point for the feature research. We show that this modification works slower than other approaches because it compares each new individual with a set of unique individuals in the population. Nevertheless, the WD and WD-T versions of the genetic solver give results with good quality in less number of generation. Better implementation of this approach could improve the results of the genetic solver.

Another starting point for the future work is a parameter control inside the genetic solver. These approaches highly recommended by many researches. 