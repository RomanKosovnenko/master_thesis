\chapter{Future work}

The research presented in this thesis shows that parameter tuning could improve genetic solver. However, it is not enough to solve big sized problems. As mentioned in Section\ref{label}, there is a framework limitation of used selector algorithms. There is a possibility that another selection algorithm could improve results, because during implementing different modification BRISE sometimes change it.

Genetic solver without duplicates (WD) is a possible started point for feature research. We show that this modification works slower than other approaches because it compares each new individual with a set of unique individuals in the population. Nevertheless, the WD and WD-T versions of the genetic solver give results with good quality in less number of generation. Better implementation of this approach could improve the results of the genetic solver.

Another started point is parameter control inside the genetic solver. It will change the parameters of the genetic solver on a fly. 