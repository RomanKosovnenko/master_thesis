\chapter{Background}
This thesis based on several kinds of research of SAS, auto-tuning, evolutionary algorithms, and software optimization. This chapter summarizes the important aspects and details of approaches and optimization technics used in the succeeding chapters.  

\section{Multi Quality Auto Tunning (MQuAT)}
Multi-Quality Auto-Tuning (MQuAT) – is an approach to self-adaptive software, which provides design and operation principles for software systems that automatically provide the best possible utility to the user while producing the least possible cost~\cite{gotz13}.
It is based on the design-time part, which represents a new development method for self-optimizing systems and runtime parts, which concerns operation principles, namely,  novel techniques to runtime self-optimization~\cite{gotz13}.

\subsection{Design principles of MQuAT}
MQuAT presented a new method of developing self-optimized software. In which software is proposed to be constructed from components with specifically defined limits. In addition, components are intended to comprise multiple implementations, each providing the same but differing functionality in their non-functional behavior. Therefore, the design principle is the critical factor for runtime optimization because when there is a different configuration, optimization can be done, and the optimal or almost optimal configuration can be selected. To compare different implementations of the software component, they need to be specified with their non-functional properties (NFP)~\cite{gotz13}.
A new meta-architecture of the self-optimizing software system was created To highlight this fundamental principle. It called the cool component model~\cite{gotz10}.
A specialty of MQuAT is the application of QoS contracts to cover non-functional implementation behavior as well as the interrelationships of different components between NFPs. Contracts naturally describe the relationship between provisions and requirements.

\subsection{Operation principles of MQuAT}
The core runtime approach proposed in MQuAT is the THE Auto-Tuning Runtime Environment (THEATRE)\cite{gotz10, gotz12}.
The concept of runtime environment contains three layers: a user, software, and a resource layer. All of them depicted in Figure ~\ref{fig:threelayersmquat}
\begin{figure}
	\centering
	\includegraphics[width=\textwidth]{images/ThreeLayersMQuAT}
	\caption[Layers of MQuAT]{}
	\label{fig:threelayersmquat}
\end{figure}

The user layer invokes features and identifies their specifications. The  Global  User  Manager  (GUM)  is required to manage the mapping between users and software component implementations and to coordinate the requests. User requests are minimal software system specifications that have to be met to satisfy the user. The general aim of the runtime system is to help the user as effectively as possible about the purposes of the user~\cite{gotz13}.

The software layer contains all software components, and it's implementations. Each Component has a Local Quality Manager(LQM), which responsible for controlling the set of components~\cite{gotz13, ahmad18}. Also, there is a single Global Quality Manager (GQM) which carry out the study and preparation phases of the feedback loop, while the LQM is responsible for the implementation process~\cite{gotz13}.

The resources layer comprises physical (e.g., CPU or RAM) as well as virtual resources(e.g., operating system).
To controlling and monitoring all resources, the Global Resource Manager(GRM) is presented. Each resource has its own Local Resource Manager (LRM), which has in-depth knowledge about its resource and the ability to steer the resource by~\cite{gotz13, ahmad18}.

\subsection{MQuAT combine the design time and runtime}
\begin{figure}
	\centering
	\includegraphics[width=\textwidth]{images/CombinedMQuAT}
	\caption[Layers of MQuAT]{}
	\label{fig:CombinedMQuAT}
\end{figure}
As showed in Figure ~\ref{fig:CombinedMQuAT}, in addition to the actual code, the developer is creating the models and quality contracts. The users interact with the system at runtime and prepare their objectives and requests. 
In addition to the running components, the runtime system includes a runtime model, representing the current state of the system. The objectives of the user are transformed into objective functions of an optimization program prior to optimization. Depending on the type of optimization technique used, either all users ' objectives are merged into a single objective function, or one objective function is extracted per purpose. Such objective functions and the system's runtime model are used by the runtime optimization method to produce formulations of the optimization problem for the respective technique. Finally, the system determines whether the optimal configuration differs from the actual configuration~\cite{gotz13, ahmad18}.

For more details about MQuAT, please read refer~\cite{gotz13}. For this thesis, we are discussing a solver of the MQuAT problem that is based on MQuAT. Hence, detailed knowledge about MQuAT is not required. 


\section{MQuAT problem}

MQuAT problem was presented in ~\cite{gotz18}, and it consists of two problems:

\begin{itemize}
	\item Resource allocation in which the mapping of software component implementation to hardware resource leads to the least cost
	\item Variant selection, which provides the best utility by selecting better software implementations.
\end{itemize} 

To solve this problem was presented new generic metamodel. Both problems are interrelated by user requests specifying minimum requirements on the provided non-functional properties (i.e., minimum utility) while searching for a selection and mapping both to maximize utility and minimize costs. Correctness denotes that only solutions that \textit{do not violate} the users minimum requirements are considered \textbf{valid}.

The problem to be solved is selecting variants of software components and mapping them based on user requests to suitable hardware resources.

The MQuAT problem could be described as a metamodel which consists of:
\begin{itemize}
	\item Hardware metamodel, which consists of hierarchically structured resource types and resources as instances of these types. So the hardware model composes static awareness of resources (types) and knowledge of runtime (instances). Certain types of resources can run the software, i.e., they are valid targets for software implementation mapping. The container attribute is used to mark such types. Figure~\ref{fig:HWmodel} depicted Hardware metamodel.
	\begin{figure}
		\centering
		\includegraphics[width=\textwidth]{images/HWModel}
		\caption{}
		\label{fig:HWmodel}
	\end{figure}
	In addition, a set of properties further characterize resource types. Resources specify then specific values for these properties. As an example, the resource type RAM could be defined with a property amount of memory and marked as a container
	\item Software metamodel showed on Figure~\ref{fig:SWModel}. Its main element called Component and represents some functionality.
	Each Component consists of implementations that provide this functionality, requiring additional components or resources to complete their work. 
	\begin{figure}
		\centering
		\includegraphics[width=\textwidth]{images/SWModel}
		\caption{}
		\label{fig:SWModel}
	\end{figure}
	\item The Objective specifies how to calculate a solution's objective value, i.e., for which value(s) the problem should be optimized. This selects a property to optimize for, and an objective function to define how to aggregate all values of this property.
	\item A Request represents a user requirement, which specifies which algorithm should be used to execute parameters and requirements. Requests contain their functional requirements by referring to a target software component, limitations on non-functional requirements (e.g., quality).
\end{itemize}
Full problem depicted on Figure~\ref{fig:mquatmodel}
\begin{figure}
	\centering
	\includegraphics[width=\textwidth]{images/MQuATModel}
	\caption{}
	\label{fig:mquatmodel}
\end{figure}

There are some constraints that are grouped in Architectural, Request, and Negotiation constraints groups.
\begin{itemize}
	\item Architectural constraints ensure that each request is fulfilled, selecting exactly one implementation per Component and deploying no more than one implementation on one resource.
	\item Request constraints ensure components are selected for each request so as to provide the requested non-functional properties.
	\item Negotiation constraints ensure non-functional requirements are met depending on the implementation.
\end{itemize}
There are additional constraints due to problem generation:
\begin{itemize}
	\item structures for the software and hardware components are fixed to ensure comparability,
	\item computeNode that represents a regular computer hardware consist of one or more CPUs, RAM memory, disk, and a networking interface,
	\item software model has a simple tree structure,
	\item fixed branching factor of two.
\end{itemize}

\todo{Should I add parameters that describe a problem(software variants, number of requests and so on)?)}

\section{The solution of the MQuAT problem}
The solution is computed by the MQuAT solver. There are many solvers:
\begin{itemize}
	\item Simple solver, which goes step by step from one solution candidate to another.
	\item ILP solver, which generates integer linear programming (ILP) problem from the MQUAT problem and after that solve it.
	\item Random solver - tries random solution candidates.
	\item Simulated Annealing (SA) solver, based on the simulated annealing meta-heuristic~\cite{pukhkaiev19}.
	\item Genetic solver that uses a genetic algorithm to solve the problem.
\end{itemize}
In this thesis, we will talk about Genetic solver in detail in the next sections.
The solution could be represented as a tree structure. An example of the solution shown in Figure ~\ref{fig:SolutionModel}
\begin{figure}
	\centering
	\includegraphics[width=\textwidth]{images/MQuATSolutionModel.png}
	\caption{}
	\label{fig:SolutionModel}
\end{figure}
It contains a list of assignments. Each assignment select one implementation of required component and map it to the resources~\cite{gotz18}.
A solution is valid if for each user request
\begin{enumerate}
	\item an implementation is deployed for the target component,
	\item for each Component required implementation is deployed,
	\item all necessary (non-functional) property clauses (including request constraints) are met,
	\item at most one implementation for each resource is deployed.
\end{enumerate}
If it is valid and no other solution has a better objective value, then a solution is optimal~\cite{gotz18}.

\section{Genetic algorithm}
\label{sec:GeneticAlgorithm}
Evolutionary algorithms are a subset of evolutionary computation and belong to set of modern heuristics-based search method.~\cite{vikhar16}
Appeared as a result of the influence of the biological evolution on computer scientists. This domain contains different types.
There are
\begin{itemize}
	\item Genetic algorithm 
	\item Genetic programming
	\item Evolutionary programming
	\item Evolution strategy
\end{itemize}

\begin{figure}
	\centering
	\includegraphics[width=\textwidth]{images/GeneticLoop}
	\caption{Main loop of the genetic algorithm}
	\label{fig:GeneticLoop}
\end{figure}
Main principle of GA looks like a loop with several steps.
The first step is the creation of an initial population - a set of randomly created individuals; each of them represents one solution. 
The second step is Evaluation. Calculate the fitness or objective function of the current population.
If the solution is founded and all requirements for the termination are fulfilled, then we get the final solution. Otherwise, select the best candidates to create a new generation.
In step 4, using recombination and mutation on selected candidates, GA creates a new generation of the population. And after that, evaluate it.

GA based on different components and operators.
\subsection{Selector}
The selector is one of the most important components of GA. It selects \textit{mu} number of individuals from the population.

There are many different selection algorithms
\begin{itemize}
	\item NSGA2 (Non-dominated sorting based genetic algorithm)
	\item SPEA2 (Strength Pareto Evolutionary Algorithm)
	\item NSGA3\todo{litref}
	\item SPEA3\todo{litref}
	\item PDE \todo{litref}
\end{itemize}
In this thesis, we will focus on a selection algorithm only as a parameter of a genetic algorithm and will use NSGA2 and SPEA2 due to software constraints of the used framework. 
\todo{Will add more information about selectors}
\subsection{crossover}
Crossover is an operator of a genetic algorithm that allows the recombination of two individuals by swapping some genes between them.
In general, the crossover has several parameters such as
\begin{itemize}
	\item Crossover rate - parameter that describes the probability of two chromosomes to exchange their genes.
	\item Crossover point - the point in which the exchange could be done.
\end{itemize}

The principle of the crossover is next.
Firstly, select the crossover point. For example, a chromosome could be described as a vector of bits. Then the crossover point is the start index of bits, which were replaced by another chromosome.
Secondly, swap genes between chromosomes.

\begin{figure}
	\centering
	\includegraphics[width=\textwidth]{images/crossoverVector.png}
	\caption{Main loop of the genetic algorithm}
	\label{fig:crossoverVector}
\end{figure}

\subsection{mutation}

The mutation is an operator of a genetic algorithm that changes a single gene in a chromosome. As a crossover operator, a mutation has parameters:

\begin{itemize}
	\item Mutation rate - parameter that describes the probability mutation.
\end{itemize}

To perform a mutation on chromosome need to do:
\begin{enumerate}
	\item Randomly select gene which mutates
	\item Change selected gene to another.
\end{enumerate}

Figure~\ref{fig:MutationVector} presented a simple mutation on the chromosome that described as a vector of bits.

\begin{figure}
	\centering
	\includegraphics[width=\textwidth]{images/MutationVector.png}
	\caption{Main loop of the genetic algorithm}
	\label{fig:MutationVector}
\end{figure}


\section{Genetic solver}
To solve the MQuAT problem using a genetic algorithm, the genetic solver was developed by Jamal Ahmad in ~\cite{ahmad18}. And further improved by Johannes Mey.

This solver is based on Opt4J framework~\footnote{http://opt4j.sourceforge.net/download.html}. It's an open-source framework that gives the opportunity to implement a genetic algorithm for custom optimization problem by specifying several modules and classes.

To solve the custom problem using genetic algorithm, the user needs to create several things:
\begin{enumerate}
	\item Creator is needed to create a random genotype for the initial population.
	In the genetic solver Creator create the genotype by creating a random solution model and transform it into a Tree Shape Genotype structure.
	\item Decoder is needed to perform decoding the tree shape genotype into phenotype. The phenotype, in this case, is a Solution Model of MQuAT.
	\item Evaluator calculates the objective functions of the solution. In the case of genetic solver, the Evaluator calculates two objectives: 
	\begin{itemize}
		\item Validity errors - number of violated contracts
		\item Energy value - energy consumption
	\end{itemize}
\end{enumerate}

If your genotype can't be described as a vector, then you first need to implement:
\begin{enumerate}
	\item Genotype
	\item Crossover operator
	\item Mutation operator
\end{enumerate}

\subsection{Tree Shape Genotype}
Because of the problem model of MQuAT, which requires mapping of implementations to resources, in genetic solver was created Tree Shape Genotype~\cite{ahmad18}.
The example of this genotype shown on Figure~\ref{fig:TreeShapeGenotypeExample}

\begin{figure}
	\centering
	\includegraphics[width=\textwidth]{images/TreeShapeGenotypeExample.png}
	\caption{}
	\label{fig:TreeShapeGenotypeExample}
\end{figure}

The first node in this genotype contains the input request. The second node represents the mapping of the user component. In this node, Impl A-1 is selected rather than Impl A-0 of the user component. As shown in Figure~\ref{fig:TreeShapeGenotypeExample}, this implementation is mapped to Hardware-Resources1. Moreover, Impl A-1 also requires software components B and C that have only one Impl B-0 and Impl C-0 implementation and are mapped to Hardware-Resource5 and Hardware-Resource4, respectively.

\subsection{Crossover operator}
This operator performs a crossover between two Tree Shaped Genotypes and are performed on software implementation and hardware resource in each tree shape genotype. Figure~\ref{fig:GeneticSolverCrossover} depicted the crossover. 
During this process, there is a check that ensures that the implementation and resources are not the same. If this check showed that comparing nodes are the same, then crossover goes recursively to child nodes and performs the check.
If the check showed that nodes are not the same, they swap the implementation, the resource, and all lower substructure.

\begin{figure}
	\centering
	\includegraphics[width=\textwidth]{images/GeneticSolverCrossover.png}
	\caption{Main loop of the genetic algorithm}
	\label{fig:GeneticSolverCrossover}
\end{figure}



\subsection{Mutation operator}
Mutation operation is used to randomly add new features to the tree-shaped genotype. Figure~\ref{fig:GeneticSolverCrossover} depicts the mutation process in which with some probability to mutate current node or recursively go down to children.
There are two more probabilities that represent a chance of mutation in implementation and resource accordingly. 

\begin{figure}
	\centering
	\includegraphics[width=\textwidth]{images/GeneticSolverMutation.png}
	\caption{Main loop of the genetic algorithm}
	\label{fig:GeneticSolverMutation}
\end{figure}

\section{Parameter Tuning Strategies for GA}
information from literature!\todo{will add more in next draft}

\section{BRISE}
BRISE is a software product line (SPL) for parameter tuning.
BRISE has a distributed structure and consists of three components.
\begin{itemize}
	\item Main Node
	\item Worker Service
	\item Worker
\end{itemize}
All these components work as docker containers and communicate between themselves using queens. 
\begin{figure}
	\centering
	\includegraphics[width=\textwidth]{images/BRISEarch.png}
	\caption{Main loop of the genetic algorithm}
	\label{fig:BRISEarch}
\end{figure}

\subsection{Main node}
\begin{figure}
	\centering
	\includegraphics[width=\textwidth]{images/BRISEMain.png}
	\caption{Main loop of the genetic algorithm}
	\label{fig:BRISEMain}
\end{figure}
Main Node is a core component of the framework, it is responsible for the whole flow of the application except the measurement process.~\cite{pukhkaiev19} 

\subsection{Worker Service}


\subsection{Worker}

The Worker is a lightweight component, containing only the logic of the algorithm to be tuned (or its wrapper), which should be specified by the user. It gets a parameter combination to be tested as an input and outputs the quality metric to be later evaluated in the Main Node. 








